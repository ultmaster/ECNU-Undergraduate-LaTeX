% !TEX root
% !TEX program = xelatex
% !BIB program = biber

% \def \PrintMode{} %在使用电子版论文时,请将此行注释。在打印纸质论文时,请保持本行命令不被注释,然后打印时选择双面打印即可。

%用来控制是否启动打印模式的宏,请勿改动。
\ifx \PrintMode \undefined
    \def \SideMode{oneside}
    \def \ClearPageStyle{\clearpage}
\else
    \def \SideMode{twoside}
    \def \ClearPageStyle{\cleardoublepage}
\fi

\documentclass[a4paper,\SideMode,UTF8]{article} %A4纸,UTF-8

\usepackage{packages_and_settings} %加载各宏包以及本模板的主要设置
\addbibresource{./reference/thesis-ref.bib} %加载bib文件(参考文献)

\begin{document}

\pagestyle{empty} %不对正文前的各页面使用页眉页脚
\newgeometry{top=2.0cm, bottom=2.0cm,left=3.18cm, right=3.18cm} %设置用于首页的页边距
\input{./preface/inner-cover.tex} %插入内封面
\ClearPageStyle

\restoregeometry
%生成目录
\addtocontents{toc}{\protect\thispagestyle{empty}}
\begin{spacing}{1}
    \tableofcontents
\end{spacing}
\ClearPageStyle
\pagenumbering{Roman}
\thispagestyle{fancy}
\newcommand{\TitleCHS}{ 题目} %中文标题

\newcommand{\TitleENG}{ Title } %英文标题

\newcommand{\Author}{张三} %作者名字

\newcommand{\StudentID}{23333333333} %学号

\newcommand{\Department}{计算机科学技术学院} %学院

\newcommand{\Major}{计算机科学与技术} %专业

\newcommand{\Supervisor}{李四} %导师名字

\newcommand{\AcademicTitle}{副工程师} %导师职称

\newcommand{\CompleteYear}{2020} %毕业年份

\newcommand{\CompleteMonth}{6} %毕业月份

\newcommand{\KeywordsCHS}{XML,XSL,数据批处理,``双赢''} %中文关键词

\newcommand{\KeywordsENG}{XML, XSL, batch, `Win-Win'} %英文关键词

\renewcommand\abstractname{\rmfamily\zihao{-3} 摘 \quad\quad 要}
\phantomsection
\begin{abstract}
	\addcontentsline{toc}{section}{摘 \quad\quad 要}
	\zihao{5}\rmfamily
	\vspace{\baselineskip}
	\par 本文通过一个实际的对日软件外包案件的设计和实现,经历了整个软件开发的过程,包括系统分析、概要设计、详细设计、编码、测试,为某制药企业开发了一个B2B的电子商务系统。本系统主要是以该制药企业为购买方,发布企业所需要的货物清单,以本系统为平台,各个供应商进行竞标,由购买方选择购买供应商的货物并下订单发货。由于购买方发布货物需要对大量数据进行操作,因此制作了数据批处理程序,来实现大量数据的导入和导出。系统的在线部分运用了XML和XSL技术,体现了画面实现模版化的优势,使得更有效方便的实现画面重载。该系统的开发过程中,本人主要负责制造在线部分的登陆、Home、估价请求履历检索、取消订单一览、用户信息作成、通知信息作成6个模块以及数据批处理部分的供应商Master导入程序。
	\par 使用该系统,制药企业可以在众多供应商中选择最价廉物美的原材料,这样大大降低其成本,提高了企业利润。同时,供应商之间也有了相互竞争,可以促进生产,达到 ``双赢'' 的效果。
	\par 本文最后说明了对日软件开发过程与当今我国软件开发过程的区别,并对我国今后软件事业做了期望和展望。
    \newline
    \newline
    \newline
	{\bfseries \rmfamily\zihao{5} 关键词:} \zihao{5}{\rmfamily \KeywordsCHS}
\end{abstract} %生成中英文摘要及关键词
\ClearPageStyle

\thispagestyle{fancy}
\renewcommand\abstractname{\zihao{-3} Abstract}
\phantomsection
\begin{abstract}
    \addcontentsline{toc}{section}{Abstract}
    \zihao{5}
    \vspace{\baselineskip}
    \par This paper concerns with design and achievement of a software development project for Japanese company, experence the whole process of software development including system analysis, general design, detail design, coding, testing to develop a B2B E-business system for some Medicine Manufacture Enterprise. In this system, the Medicine Manufacture Enterprise as the buyers publish products list.The suppliers use this system as plat to bit products.Then the buyers select suppliers they need and send order to buy products.This system also has some batches to realize huge data importing and exporting.This system’s webs use XML and XSL technologies to realize reloading of web templates. During the development of the system, my duty is to develop six webs and one batch.
	\par Using this system, Medicine Manufacture Enterprise can select the best and the cheapest products in several suppliers to descreat cost. At the same time, suppliers will bit each other to inprove production and reach ‘Win-Win’.
	\par At the end of this paper, also explain the deference between Janpanese software development processing and Chinese, and give the deep expectance.
    \newline
    \newline
    \newline
    {\bfseries \zihao{5} Keywords:} {\zihao{5} \KeywordsENG}
\end{abstract} %生成中英文摘要及关键词
\ClearPageStyle
\pagenumbering{arabic}
\pagestyle{fancy} %开始使用页眉页脚
\setcounter{page}{1} %论文页码从正文开始记数

\input{./body/SectionA.tex} %正文第一章
\input{./body/SectionB.tex} %正文第二章
\input{./body/SectionC.tex} %正文第三章
\section{表与图}这节用来展示表格与图片的插入。

\subsection{表格}
\par 本来LaTeX里表格的变化是非常多的,但鉴于学校要求用三线式,问题反而简单了。以下是一个例子:
\begin{table}[htbp]\center
    \bicaption{示例表格}{Example Table}
    \begin{tabular}{lcccccl}
     \toprule
     。。 & 。。 & 。。 & 。。 & 。。& 。。 & 。。\\
     \midrule
    。。 & 。。 & 。。 & 。。 & 。。& 。。 & 。。\\
    。。 & 。。 & 。。 & 。。 & 。。& 。。 & 。。\\
    。。 & 。。 & 。。 & 。。 & 。。& 。。 & 。。\\
    。。 & 。。 & 。。 & 。。 & 。。& 。。 & 。。\\
    。。 & 。。 & 。。 & 。。 & 。。& 。。 & 。。\\
     \bottomrule
    \end{tabular}
   \end{table}
如果你有使用更复杂的表格的需求,请自行查资料完成。

\subsection{插图}
由于这份模板不考虑多栏排版,所以格式要求中所述的半栏图大小要求我们不作演示。以下是一个通栏图的演示:
\begin{figure}[H]
    \centering
    \includegraphics[width=100mm]{example-image}
    \bicaption{图片测试(最小宽度)}{Image test (Minimal width)}
  \end{figure}

\begin{figure}[H]
    \centering
    \includegraphics[width=130mm]{example-image}
    %\includegraphics[width=130mm]{./figures/你自己的图像文件}
    \bicaption{图片测试(最大宽度)}{Image test (Maximal width)}
\end{figure}
\par 注意:这里为了减少图片上下的空白,使用了float宏包。 %正文第四章
\input{./body/SectionE.tex} %正文第五章

\theendnotes %尾注(若没有尾注请将本行删除)
\ClearPageStyle

%生成参考文献
\phantomsection
\addcontentsline{toc}{section}{参考文献}
\printbibliography[title={\centerline{\bfseries\sffamily \zihao {-3}参考文献}}]
\ClearPageStyle

%生成附录
\phantomsection
\addtocontents{toc}{\setcounter{tocdepth}{1}}
\addcontentsline{toc}{section}{附 \quad\quad 录}
\setcounter{subsection}{0}
\ctexset { subsection = { name={,},number={\arabic{subsection}},format={\rmfamily \zihao {5}} } }
\ctexset { subparagraph = { name={(,)},number={\arabic{subparagraph}},format={\rmfamily \zihao {5}},indent=2em } }
\section*{\centerline{\bfseries \sffamily \zihao{-3} 附 \quad\quad 录}}

\subsection{实验数据}
\subparagraph{吐槽}
2019年的样板做得实在太烂了

\subsection{调查结果}
23333333333333333333333333333333333333333
\ClearPageStyle

\makeacknowledgement %生成感谢

\end{document} 
