\thispagestyle{fancy}
\newcommand{\TitleCHS}{ 题目} %中文标题

\newcommand{\TitleENG}{ Title } %英文标题

\newcommand{\Author}{张三} %作者名字

\newcommand{\StudentID}{23333333333} %学号

\newcommand{\Department}{计算机科学技术学院} %学院

\newcommand{\Major}{计算机科学与技术} %专业

\newcommand{\Supervisor}{李四} %导师名字

\newcommand{\AcademicTitle}{副工程师} %导师职称

\newcommand{\CompleteYear}{2020} %毕业年份

\newcommand{\CompleteMonth}{6} %毕业月份

\newcommand{\KeywordsCHS}{XML,XSL,数据批处理,``双赢''} %中文关键词

\newcommand{\KeywordsENG}{XML, XSL, batch, `Win-Win'} %英文关键词

\renewcommand\abstractname{\sffamily\zihao{-3} 摘 \quad\quad 要}
\phantomsection
\begin{abstract}
	\addcontentsline{toc}{section}{摘 \quad\quad 要}
	\zihao{5}\rmfamily
	\vspace{\baselineskip}
	\par 本文通过一个实际的对日软件外包案件的设计和实现,经历了整个软件开发的过程,包括系统分析、概要设计、详细设计、编码、测试,为某制药企业开发了一个B2B的电子商务系统。本系统主要是以该制药企业为购买方,发布企业所需要的货物清单,以本系统为平台,各个供应商进行竞标,由购买方选择购买供应商的货物并下订单发货。由于购买方发布货物需要对大量数据进行操作,因此制作了数据批处理程序,来实现大量数据的导入和导出。系统的在线部分运用了XML和XSL技术,体现了画面实现模版化的优势,使得更有效方便的实现画面重载。该系统的开发过程中,本人主要负责制造在线部分的登陆、Home、估价请求履历检索、取消订单一览、用户信息作成、通知信息作成6个模块以及数据批处理部分的供应商Master导入程序。
	\par 使用该系统,制药企业可以在众多供应商中选择最价廉物美的原材料,这样大大降低其成本,提高了企业利润。同时,供应商之间也有了相互竞争,可以促进生产,达到 ``双赢'' 的效果。
	\par 本文最后说明了对日软件开发过程与当今我国软件开发过程的区别,并对我国今后软件事业做了期望和展望。
    \newline
    \newline
    \newline
	{\bfseries \rmfamily\zihao{5} 关键词:} \zihao{5}{\rmfamily \KeywordsCHS}
\end{abstract}